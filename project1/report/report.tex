\documentclass[english,notitlepage,aps,pra,10pt]{revtex4-2}
%For preview: skriv i terminal: latexmk -pdf -pvc filnavn

% if you want a single-column, remove reprint

% allows special characters (including æøå)
%\usepackage[utf8]{inputenc}
%\usepackage[english]{babel}

%% note that you may need to download some of these packages manually, it depends on your setup.
%% I recommend downloading TeXMaker, because it includes a large library of the most common packages.

\usepackage{physics,amssymb}  % mathematical symbols (physics imports amsmath)
\usepackage{amsmath}
% \usepackage{amssymb}
\usepackage{graphicx}         % include graphics such as plots
\usepackage{xcolor}           % set colors
\usepackage{hyperref}         % automagic cross-referencing (this is GODLIKE)
\usepackage{listings}         % display code
\usepackage{subfigure}        % imports a lot of cool and useful figure commands
\usepackage{float}
%\usepackage[section]{placeins}
\usepackage{algorithm}
\usepackage[noend]{algpseudocode}
\usepackage{subfigure}
\usepackage{tikz}
\usetikzlibrary{quantikz}
% defines the color of hyperref objects
% Blending two colors:  blue!80!black  =  80% blue and 20% black
\hypersetup{ % this is just my personal choice, feel free to change things
    colorlinks,
    linkcolor={red!50!black},
    citecolor={blue!50!black},
    urlcolor={blue!80!black}}

%% Defines the style of the programming listing
%% This is actually my personal template, go ahead and change stuff if you want



%% USEFUL LINKS:
%%
%%   UiO LaTeX guides:        https://www.mn.uio.no/ifi/tjenester/it/hjelp/latex/
%%   mathematics:             https://en.wikibooks.org/wiki/LaTeX/Mathematics

%%   PHYSICS !                https://mirror.hmc.edu/ctan/macros/latex/contrib/physics/physics.pdf

%%   the basics of Tikz:       https://en.wikibooks.org/wiki/LaTeX/PGF/Tikz
%%   all the colors!:          https://en.wikibooks.org/wiki/LaTeX/Colors
%%   how to draw tables:       https://en.wikibooks.org/wiki/LaTeX/Tables
%%   code listing styles:      https://en.wikibooks.org/wiki/LaTeX/Source_Code_Listings
%%   \includegraphics          https://en.wikibooks.org/wiki/LaTeX/Importing_Graphics
%%   learn more about figures  https://en.wikibooks.org/wiki/LaTeX/Floats,_Figures_and_Captions
%%   automagic bibliography:   https://en.wikibooks.org/wiki/LaTeX/Bibliography_Management  (this one is kinda difficult the first time)
%%   REVTeX Guide:             http://www.physics.csbsju.edu/370/papers/Journal_Style_Manuals/auguide4-1.pdf
%%
%%   (this document is of class "revtex4-1", the REVTeX Guide explains how the class works)


%% CREATING THE .pdf FILE USING LINUX IN THE TERMINAL
%%
%% [terminal]$ pdflatex template.tex
%%
%% Run the command twice, always.
%% If you want to use \footnote, you need to run these commands (IN THIS SPECIFIC ORDER)
%%
%% [terminal]$ pdflatex template.tex
%% [terminal]$ bibtex template
%% [terminal]$ pdflatex template.tex
%% [terminal]$ pdflatex template.tex
%%
%% Don't ask me why, I don't know.

\newcommand{\e}{\mathrm{e}}
\newcommand{\bv}[1]{\mathbf{#1}}


\begin{document}

\title{FYS4150 - Project 1}      % self-explanatory
\author{Halvor Melkild}          % self-explanatory
\date{\today}                             % self-explanatory
\noaffiliation                            % ignore this, but keep it.


\maketitle

\textit{GitHub repo: \href{https://github.com/halvorme/FYS4150/}{https://github.com/halvorme/FYS4150}}

\section*{Problem 1}

We are looking at the one-dimensional Poisson equation 
\begin{equation}
    -\dv[2]{u}{x} = f(x),
    \label{eq:poisson}
\end{equation}
on the range $x \in [0,1]$. The boudary conditions are $u(0) = u(1) = 0$ and source term is chosen as 
\begin{equation}
    f(x) = 100\, \e^{-10x}.
\end{equation}
An exact solution to Equation~\ref{eq:poisson} has the form 
\begin{equation}
    u(x) = -\e^{-10x} + C_1 x + C_2.
\end{equation}
The boundary condition at $x=0$ gives that $C_2 = 1$, and the condition at $x=1$ then gives that $C_1 = \e^{-10} - 1$. It follows that 
\begin{equation}
    u(x) = 1 - (1-\e^{-10})x - \e^{-10x} 
\end{equation}
is a solution to Equation~\ref{eq:poisson}, satisfying the given boundary conditions.


\section*{Problem 2}

The plot of the exact solution is seen in Figure~\ref{fig:uExact}. The generating code is found in the repository linked at the top of the document.

\begin{figure}%[h!]
    \begin{center}
        \includegraphics{../imgs/u_exact.pdf}
        \caption{The plot shows the exact solution to the Poisson equation (Eq.~\ref{eq:poisson}), with the source term $f(x) = 100 \e^{-10x}$, and boundary conditions $u(0)=u(1)=0$.}
        \label{fig:uExact}     
    \end{center}
\end{figure}

\section*{Problem 3}

To make a discretised version of the Poisson equation, we need to discretise the second derivative. We start out with the definition of the derivative
\begin{equation}
    u'(x) = \dv{u}{x} = \lim_{\Delta x \to 0} \frac{u(x+\Delta x) - u(x)}{\Delta x}.
\end{equation}
If a function is differentiable at a point $x$ it means that the limit has to be the same if you approach $x$ from above or below. It follows that 
\begin{equation}
    u'(x) = \lim_{\Delta x \to 0} \frac{u(x) - u(x-\Delta x)}{\Delta x}
\end{equation}
is an equivalent definition. The second derivative can then be defined the following way,
\begin{equation}
    \begin{split}
        u''(x) = \dv[2]{u}{x} &= \lim_{\Delta x \to 0} \frac{u'(x+\Delta x) - u'(x)}{\Delta x} \\
            &= \lim_{\Delta x \to 0} \frac{u(x+\Delta x) - 2u(x) + u(x-\Delta x)}{\Delta x^2}.
    \end{split}
    \label{eq:secDer}
\end{equation}
When we discretise the $x$-axis it will no longer be possible to take the limit $\Delta x \to 0$. The smallest possible value is $\Delta x = h$, where $h$ is the stepsize of the discretisation. We will than have to find an approximation for $u''(x)$. By Taylor expanding 

We would now like to discretise this expression. We start by discretising the $x$-axis. The line segment $[1,0]$ is replaced by the set with $N+1$ elements, 
\begin{equation}
    \{x_i = i h \mid i\in\{0,1, \dots, N\},\ h = \frac{1}{N}\}.
\end{equation}
The distance between each point is given by $h$. We notice that $x_i + nh = x_{i+n}$ for any integer $n$. The function $u(x)$ is now replaced by the set
\begin{equation}
    v_i = u(x_i),
\end{equation}
where we choose denote the solution to the discretised equation as $v$. The boundary conditions tells us that $v_0 = v_N = 0$. For the source term we will simply use $f_i = f(x_i)$. 

When the function $u(x)$ is discretised, we can't take the limit $\Delta x \to 0$ any more. The best we can do is $\Delta x \to h$. We than has to work with an approximation to Equation~\ref{eq:secDer}. By Taylor expanding 
\begin{equation}
    u(x \pm h) = u(x) \pm h u'(x) + \frac{1}{2} h^2 u''(x) \pm \frac{1}{6} h^3 u^{(3)}(x) + \frac{1}{24} h^4 u^{(4)} + \mathcal{O}(h^5)
\end{equation}
we find that 
\begin{equation}
    u''(x) = \frac{u(x+h) - 2u(x) + u(x-h)}{h^2} + \mathcal{O}(h^2).
\end{equation}
Discarding the terms of $\mathcal{O}(h^2)$ and discretising the remainding expression we get
\begin{equation}
    v_i'' = \frac{1}{h^2}(v_{i+1} - 2v_i + v_{i-1}).
\end{equation}

In the end, the discretised Poisson equation (Eq.~\ref{eq:poisson}) can be written as
\begin{equation}
    -v_{i-1} + 2v_i - v_{i+1} = h^2 f_i,
    \label{eq:discPoisson}
\end{equation} 
where $i \in \{1,\dots, N-1\}$.

\section*{Problem 4}

The discretised Poisson equation (Eq.~\ref{eq:discPoisson}) gives us a set of $N-1$ equations, with $N-1$ unknowns, as $v_0$ and $v_N$ are set by the boundary conditions. We can write each of these equation on matrix form
\begin{equation}
    \begin{pmatrix}
        -1 & 2 & -1
    \end{pmatrix}
    \begin{pmatrix}
        v_{i-1} \\
        v_i \\
        v_{i+1}
    \end{pmatrix}
    = h^2 f_i.
\end{equation} 
For the first and last equation we can use the boundary conditions and simplify them as  
\begin{equation}
    \begin{split}
        \begin{pmatrix}
            2 & -1
        \end{pmatrix}
        \begin{pmatrix}
            v_1 \\
            v_2
        \end{pmatrix}
        &= h^2 f_1, \\
        \begin{pmatrix}
            -1 & 2
        \end{pmatrix}
        \begin{pmatrix}
            v_{N-2} \\
            v_{N-1}
        \end{pmatrix}
        &= h^2 f_{N-1}.
    \end{split}
\end{equation}
Stacking all these equations in one matrix we get 
\begin{equation}
    \begin{pmatrix}
         2 & -1 &  0 &  0 & \cdots &  0 &  0 \\
        -1 &  2 & -1 &  0 & \cdots &  0 &  0 \\
         0 & -1 &  2 & -1 & \cdots &  0 &  0 \\
         0 &  0 & -1 &  2 & \cdots &  0 &  0 \\
        \vdots & \vdots & \vdots & \vdots & \ddots & \vdots  & \vdots\\
         0 &  0 &  0 &  0 & \cdots &  2 & -1 \\
         0 &  0 &  0 &  0 & \cdots & -1 &  2 \\
    \end{pmatrix}
    \begin{pmatrix}
        v_1 \\
        v_2 \\
        v_3 \\
        v_4 \\
        \vdots \\
        v_{N-2} \\
        v_{N-1}
    \end{pmatrix}
    =
    h^2
    \begin{pmatrix}
        f_1 \\
        f_2 \\
        f_3 \\
        f_4 \\
        \vdots \\
        f_{N-2} \\
        f_{N-1}
    \end{pmatrix}.
    \label{eq:discPoissonMat}
\end{equation}
By discretising the Poisson equation we could rewrite it as matrix equation with the form 
\begin{equation}
    A \bv{v} = \bv{g},
\end{equation}
where $A$ is a tridiagonal matrix, $\bv{v}$ is a vector of the unknown $v_i$'s and $\bv{g}$ encodes the information of the source term, but is scaled by a factor $h^2$. 


\section*{Problem 5}

The vector $\bv{v}$, as defined above, includes only the subset of internal points of $v$, which are unknown. The complete solution $\bv{v}^*$, with length $m$, also includes the two endpoints, $v_0$ and $v_N$, which are determined by the boundary conditions. That means
\begin{equation}
    m = n + 2 = N + 1.
\end{equation}


\section*{Problem 6}
\subsection*{Problem a}
We want an algorithm to solve the general matrix equation 
\begin{equation}
    A \bv{v} = \bv{g},
    \label{eq:AvEqg}
\end{equation}
where $A$ is a $n \times n$ tridiagonal matrix. We denote the subdiagonal, main diagonal and superdiagonal as $\bv{a}$, $\bv{b}$ and $\bv{c}$, respectively. The vectors $\bv{v}$, $\bv{g}$ and $\bv{b}$ has length $n$, and the off-diagonal vectors $\bv{a}$ and $\bv{c}$ has length $n-1$. 

To solve Equation~\ref{eq:AvEqg} we simply us Gaussian elimination. The algorithm is separated in forward and backward substitution. In the forward substitution, the subdiagonal $\bv{a}$ is eliminated, while $\bv{b}$ and $\bv{g}$ are modified. In the backward substitution, we solve each row for $v_i$, one after the other. The exact steps are presented in Algorithm~\ref{algo:triDiagGauss}.

\begin{algorithm}[H]
    \caption{Gaussian elimination of a tridiagonal matrix}\label{algo:triDiagGauss}
    \begin{algorithmic}
        \State Initialise $\bv{a}$, $\bv{b}$, $\bv{c}$ and $\bv{g}$ with their given values
        \State Initialise new vector $\bv{v}$ of length $n$ 

        \Comment Forward substitution
        \For{$i = 1, 2, \dots, n-1$}                    \Comment $n-1$ iterations
            \State $d = a_i/b_i$                        \Comment 1 FLOP
            \State $b_{i+1} = b_{i+1} - d\, c_i$        \Comment 2 FLOPs 
            \State $g_{i+1} = g_{i+1} - d\, g_i$        \Comment 2 FLOPs 
        \EndFor
        \Comment Backward substitution
        \State $v_n = g_n/b_n$                          \Comment 1 FLOP 
        \For{$i = 1, 2, \dots, n-1$}                    \Comment $n-1$ iterations
            \State $v_{n-i} = (g_{n-i} - c_{n-i} v_{n-i+1})/b_{n-i}$  \Comment 3 FLOPs
        \EndFor
    \end{algorithmic}
\end{algorithm}

\subsection*{Problem b}

In the right coloumn of Algorithm~\ref{algo:triDiagGauss} we have listed the number of floating-point operations (FLOPs) needed for each step. We see that there are $5(n-1)$ FLOPs needed for the forward substitution and $3(n-1)+1$ for backwards substitution. When we sum these, we find that the algorithm needs
\begin{equation}
    (8n-7) \textrm{ FLOPs}.
\end{equation} 
We note that the number of FLOPs depends linearly on $n$.


\section*{Problem 7}

The alogrithm is implemented in the function \verb+genAlgo+, found in the file \verb+algo.cpp+ in the repo. The resulting plot is found in Figure~\ref{fig:v}.

\begin{figure}%[h!]
    \begin{center}
        \includegraphics{../imgs/v.pdf}
        \caption{The plot shows the exact solution $u(x)$ to the Poisson equation and the numerical approximation $v(x)$ for three different stepsizes.}
        \label{fig:v}
    \end{center}
\end{figure}


\section*{Problem 8}


\section*{Problem 9}

\subsection*{Problem a \& b}

The algorithm presented in Problem 6 can be specialised for the purpose of the discretised Poisson equation (Eq.~\ref{eq:discPoissonMat}). As we know that $a_i = c_i = -1$ and $b_i = 2$, Algorithm~\ref{algo:triDiagGauss} can be simplified a bit. The result is shown in Algorithm~\ref{algo:PoissonGauss}. The specialised algorithm has
\begin{equation}
    (6n-5) \textrm{ FLOPs}.
\end{equation}

\begin{algorithm}[H]
    \caption{Gaussian elimination of Poisson matrix}\label{algo:PoissonGauss}
    \begin{algorithmic}
        \State Initialise $a = c = -1$, $b = 2$
        \State Initialise $\bv{g}$ with its given values
        \State Initialise new vector $\bv{v}$ of length $n$ 

        \Comment Forward substitution
        \For{$i = 1, 2, \dots, n-1$}                        \Comment $(n-1)$ iterations
            \State $b_{i+1} = 2 - 1/b_i$                    \Comment 2 FLOPs 
            \State $g_{i+1} = g_{i+1} + g_{i}/b_{i}$        \Comment 2 FLOPs 
        \EndFor
        \Comment Backward substitution
        \State $v_n = g_n/b_n$                              \Comment 1 FLOP 
        \For{$i = 1, 2, \dots, n-1$}                        \Comment $(n-1)$ iterations
            \State $v_{n-i} = (g_{n-i} + v_{n-i+1})/b_{n-i}$  \Comment 2 FLOPs
        \EndFor
    \end{algorithmic}
\end{algorithm}

% It is possible to reduce the number of FLOPs even more by noticing that the expression for $b'_i$ creates the sequence 
% \begin{equation}
%     \bv{b}' = \{2, \frac{3}{2}, \frac{4}{3}, \frac{5}{4}, \dots, \frac{n+1}{n}\}.
% \end{equation}
% As a consequence, we can write 
% \begin{equation}
%     b'_{i+1} = \frac{i+2}{i+1}
% \end{equation}

\section*{Problem 10}


\end{document}
